\chapter{Origine et nature du projet}\label{chapter:origine}

Nessa secão são descritos os objetivos da aula prática realizada. Basta falar em um parágrafo, com poucas linhas e de forma sucinta (os detalhes vem na próxima seção). E quanto aos objetivos específicos, eles são descritos na parte abaixo, utilizando o ambiente ``\textit{$\backslash$itemize\{\}}''.

Observem que a lista de objetivos específicos vai ficar em formato de tópicos. Caso fosse desejado o formato de lista numérica, poderia trocar o ``\textit{$\backslash$itemize\{\}}'' por ``\textit{$\backslash$enumerate\{\}}''.

Ou, caso deseje, pode fazer em texto corrido mesmo.

Antes que me esqueça, não precisa colocar textos após a lista de objetivos específicos. Sendo assim, pode seguir para a próxima seção.

\begin{itemize}
    \item Objetivo específico 1;
    \item Objetivo específico 2;
    \item Objetivo específico 3;
    \item Objetivo específico 4;
    \item Objetivo específico 5.
\end{itemize}

\noindent\par

