\chapter{Objet de l’étude}\label{chapter:objet}

Nessa seção você escreve sobre a teoria, materiais utilizados, etc... Deverá abordar os materiais utilizados nas aulas práticas, bem como, os equipamentos. Por exemplo:
\begin{itemize}
    \item Material utilizado na prática: liga metálica, material compósito;
    \item Tipo de corpo de prova, geometria, dimensão;
    \item Equipamento usado para a prática: máquina de ensaio, forno, termopar, cadinho, entre outros
    \item  Parâmetros para execução da prática: temperatura de ensaio, carga utilizada, etc...
\end{itemize}

Além da abordagem sobre os materiais e equipamentos, o procedimento experimental utilizado deverá ser descrito na íntegra. E, para isso, possivelmente será necessário o uso de equações.

Em LaTeX, as equações são mais chatinhas de se fazer, mas quando pegar o jeito acaba ficando fácil. Antes da equação é interessante citar a fonte, enquanto que após a equação é interessante descrever as variáveis que a compõem.

Um exemplo de equação, apresentada no trabalho de \citeonline{andrade2016}, pode ser observada abaixo.

\begin{equation}
\frac{I_g}{I_o} = C_1^{am^{C_2}}
\label{eq:e1}
\end{equation}

Onde Ig é a irradiância glocal horizontal, Io é a irradiância extraterrestre (ambas em W/m²) e am é a massa de ar.

Observe que a estrutura do parágrafo da equação \ref{eq:e1} apresenta uma breve explicação sobre as variáveis, autor(es), etc.. Recomendo ver material na internet sobre como se escreve equações em LaTeX, visto que fazer isso por aqui ia demorar bastante. Um material que recomendo é a playlist do Prof. Dr. Ygo Neto Batista. Ensina muita coisa sobre LaTeX. Você pode acessar essa playlist clicando \href{https://www.youtube.com/playlist?list=PLXQryIxlA5TAuSzx0IhXpue_8a3htOm8v}{\underline{\textit{\textbf{aqui}}}}. Lembrando que ele utiliza alguns modelos diferentes desse para ensinar LaTeX, então alguns comandos podem servir nesse template, visto que alguns dos packages utilizados são iguais.